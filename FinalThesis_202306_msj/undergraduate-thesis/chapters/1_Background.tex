\chapter{绪论}

\section{研究背景}

\space 近年来,随着经济社会的飞速发展,我国全面跨入汽车社会,汽车出行成为交通常态。据公安部统计,目前,我国驾驶人数量占成年人数量近 50%,平均每 2 个成年人中即有 1 人持有驾驶证。与此同时,汽车保有量同步迅猛增长,全国汽车保有量超过 3 亿辆,平均每百户家庭拥有汽车达到 60 辆\cite{驾驶人数据}。汽车走进千家万户,成为了普通家庭出行的常用选择之一。驾驶技能已经变成了家家户户的基本生活技能,带给交通道路的车流压力也与日俱增。

\space 在``出行难”问题横亘的当下,面对出行限号、停车位少等现实问题,共享单车、公交地铁、出租车等成为人们出行的首要选择。较共享单车来说,出租车不受车辆本身实物的限制;较公交地铁来说,出租车更加灵活便捷,能更快速地完成人们出行的需求。

\space 在传统的出租车出行领域中,存在着乘客等候时间长、司机挑客绕路、乘车环境差等问题\cite{赵杨2022网约车冲击下出租车行业转型对策研究}。随着互联网的不断发展,网络平台公司将车辆接入互联网平台,用户通过使用网络平台公司提供的手机软件,预约车辆,实现点对点运输,并对车辆进行评价反馈。网约车的兴起,极大满足了乘客个性化、便捷化的出行需求。

\space 车载网是一种特殊的移动自组织网络,它通过汽车收集、共享信息,实现更加智能、安全的驾驶\cite{程刚2011车联网现状与发展研究}。安全性和可靠性是车载网关注的主要问题\cite{Zhaojun2018A}。区块链技术具有去中心化、时序数据、集体维护、可编程和安全可信等特点\cite{袁勇2016区块链技术发展现状与展望}。因此,区块链满足分布式数据存储的需求,能为车载网数据传输的安全性、可追溯性提供技术保障、隐私保护。将区块链技术和车载网工作融合,形成一个基于区块链的出租车调度系统。区块链技术中数据信息不可篡改,因此应用进网约车系统中可以提升通讯的安全性,保证平台无法垄断打车数据,保证信息公开透明。司机与乘客之间进行直接交流与贸易,保证了司乘双方的可信交流。车与车之间进行位置通讯,也能提高通讯传输效率,营造安全高效的打车服务。

\section{相关工作}

\subsection{区块链和智能合约}

\space 在互联网上进行的电子支付活动大多数都依赖于可信第三方,但这种方式仍存在一定的风险。完全不可逆的交易不现实,调解纠纷导致的调解成本也会增加交易成本,从而使小额随机交易无法进行。因此,需要一种能允许任意两方直接交易的基于加密证明的机制,来更新现有的依赖可信第三方的电子支付方式。

\space 中本聪在2008年提出区块链技术,其在文献中描述区块链为按照时间顺序将数据区块以链条的方式组合成特定数据结构,并以密码学方式保证的不篡改和不可伪造的去中心化共享总账\cite{2008Bitcoin}。区块链采用纯数学的方法,无需可信第三方,也不牺牲隐私性,对过去、当前的数字事件建立分布式的一致性表达\cite{袁勇2016区块链技术发展现状与展望}。区块链也提供了可编程的脚本系统,增强了区块链应用的灵活性\cite{贺海武2018基于区块链的智能合约技术与应用综述}。

区块链技术目前已应用发展了三个阶段\cite{swan2015blockchain}:

\begin{enumerate}
    \item 区块链1.0:可编程货币,以比特币为代表,是数字货币应用。
    \item 区块链2.0:可编程金融,以智能合约为代表,将数字货币与智能合约结合,主要在金融领域中应用。
    \item 区块链3.0:可编程社会,为各行各业提供去中心化应用、去中心化自组织等解决方案。
\end{enumerate}

智能合约在1994年首次被Nick Szabo提出\cite{1997Formalizing}。它被定义为``一套以数字形式指定的承诺,包括合约参与方可以在其上执行这些承诺的协议”。2013年12月,Vitalik Buterin提出了以太坊区块链平台,提供了可编写智能合约的图灵完备的编程语言,才首次将智能合约与区块链技术相结合,并投入应用\cite{2014ASmartContract}。

智能合约是部署在区块链上的程序。它是可自动执行的脚本,一旦满足预先设定的条件,就可以立刻自动运行,从而使参与者不需要任何机构介入,就能获取合约运行结果。以太坊通过配备的智能合约实现任何人都可以创建的去中心化应用程序,吸引了大量开发者投入此平台,促进了一系列新型产业发展。以太坊的出现促使区块链与智能合约不再局限于数字货币,从而将区块链与智能合约的应用推广到金融领域和其他社会领域\cite{欧阳丽炜2019智能合约}。智能合约的代码语句易懂,当预定条件满足且完成验证后,区块链将执行相应行为,如释放资金、发出凭证,并在交易完成时,更新区块链。本文课题基于以太坊部署智能合约,并在此基础上,进行出租车调度系统的深入研究。

\subsection{Geohash编码技术}

移动用户的增加,提升了交互式信息地图的服务需求,促进了矢量数据地图的兴起。传统的矢量地图有Google Maps\cite{2015ggmap},它采用二维的经纬度数据表示一个唯一的地理位置。Geohash是一类新型的地址编码方式,它的编码过程如下:将经纬度分别转换成二进制字符串,将这一对二进制字符串按位交叉,奇数位为纬度序列,偶数位为经度序列,从而得到一组新字符串,将新字符串转换为十进制,并使用Base32编码方式处理该十进制字符串,从而获得该地理位置对应的Geohash编码字符串\cite{2018HGeoHashBase}。

将道路``北蜂窝路”中的经纬度(116.3227718,39.9037387)使用Geohash编码技术编码为wx4enbr8jh1的一部分过程如附录中的表\ref{geohash-weidu}和表\ref{geohash-jingdu}。

\iffalse
% 设置三线表
\begin{table}[ht]
    \linespread{1.5}
    \zihao{5}
    \centering
    \caption{纬度的二进制表示过程}
    \label{geohash-weidu}
    \begin{tabularx}{\textwidth}{XXXc}
    \toprule
    纬度 & 划分区间0 & 划分区间1 & 39.9037387 \\
    \hline
    (-90,90) & (-90,0) & (0,90) & 1 \\
    (0,90) & (0,45) & (45,90) & 0 \\
    (0,45) & (0,22.5) & (22.5,45) & 1 \\
    (22.5,45) & (22.5,33.75) & (33.75,45) & 1 \\
    (33.75,45) & (33.75,39.375) & (39.375,45) & 1 \\
    (39.375,45) & (39.375,42.1875) & (42.1875,45) & 0 \\
    (39.375,\newline 42.1875) & (39.375,\newline 40.78125) & (40.78125,\newline 42.1875) & 0 \\
    (39.375,\newline 40.78125) & (39.375,\newline 40.078125) & (40.078125,\newline 40.78125) & 0 \\
    (39.375,\newline 40.078125) & (39.375,\newline 39.7265625) & (39.7265625,\newline 40.078125) & 1 \\
    (39.7265625,\newline 40.078125) & (39.7265625,\newline 39.90234375) & (39.90234375,\newline 40.078125) & 1 \\
    (39.90234375,\newline 40.078125) & (39.90234375,\newline 39.990234375) & (39.990234375,\newline 40.078125) & 0 \\
    (39.90234375,\newline 39.990234375) & (39.90234375,\newline 39.9462890625) & (39.9462890625,\newline 39.990234375) & 0 \\
    (39.90234375,\newline 39.9462890625) & (39.90234375,\newline 39.92431640625) & (39.92431640625,\newline 39.9462890625) & 0 \\
    (39.90234375,\newline 39.92431640625) & (39.90234375,\newline 39.913330078125) & (39.913330078125,\newline 39.92431640625) & 0 \\
    (39.90234375,\newline 39.913330078125) & (39.90234375,\newline 39.9078369140625) & (39.9078369140625,\newline 39.913330078125) & 0 \\
    \bottomrule
    \end{tabularx}
\end{table}
% 结束设置三线表

% 设置三线表
\begin{table}[ht]
    \linespread{1.5}
    \zihao{5}
    \centering
    \caption{经度的二进制表示过程}
    \label{geohash-jingdu}
    \begin{tabularx}{\textwidth}{XXXc}
    \toprule
    纬度 & 划分区间0 & 划分区间1 & 116.3227718 \\
    \hline
    (-180,180) & (-180,0) & (0,180) & 1\\
    (0,180) & (0,90) & (90,180) & 1\\
    (90,180) & (90,135) & (135,180) & 0\\
    (90,135) & (90,112.5) & (112.5,135) & 1\\
    (112.5,135) & (112.5,123.75) & (123.75,135) & 0\\
    (112.5,123.75) & (112.5,118.125) & (118.125,123.75) & 0\\
    (112.5,118.125) & (112.5,115.3125) & (115.3125,118.125) & 1\\
    (115.3125,\newline 118.125) & (115.3125,\newline 116.71875) & (116.71875,\newline 118.125) & 0\\
    (115.3125,\newline 116.71875) & (115.3125,\newline 116.015625) & (116.015625,\newline 116.71875) & 1\\
    (116.015625,\newline 116.71875) & (116.015625,\newline 116.3671875) & (116.3671875,\newline 116.71875) & 0\\
    (116.015625,\newline 116.3671875) & (116.015625,\newline 116.19140625) & (116.19140625,\newline 116.3671875) & 1\\
    (116.19140625,\newline 116.3671875) & (116.19140625,\newline 116.279296875) & (116.279296875,\newline 116.3671875) & 1\\
    (116.279296875,\newline 116.3671875) & (116.279296875,\newline 116.3232421875) & (116.3232421875,\newline 116.3671875) & 0\\
    (116.279296875,\newline 116.3232421875) & (116.279296875,\newline 116.30126953125) & (116.30126953125,\newline 116.3232421875) & 1\\
    (116.30126953125,\newline 116.3232421875) & (116.30126953125,\newline 116.312255859375) & (116.312255859375,\newline 116.3232421875) & 1\\
    \bottomrule
    \end{tabularx}
\end{table}
% 结束设置三线表
\fi

通过对经纬度不断进行二分比对操作,得到``北蜂窝路”的纬度二进制序列为101110001100000,经度二进制序列为110100101011011,按照纬度为奇数位,经度为偶数位的方式,将经纬度序列排列成一个二进制序列为111001110100100011011010001010,每五位转为十进制,得到数字28,29,4,13,20,10,用表\ref{base32}中的Base32编码进行转换后得到wx4enb。
% 设置三线表
\begin{table}[ht]
    \linespread{1.5}
    \zihao{5}
    \centering
    \caption{Base32编码转换}
    \label{base32}
    \begin{tabular}{ccccccccccccccccc}
    \toprule
    Demical & 0 & 1 & 2 & 3 & 4 & 5 & 6 & 7 & 8 & 9 & 10 & 11 & 12 & 13 & 14 & 15 \\
    Base32 & 0 & 1 & 2 & 3 & 4 & 5 & 6 & 7 & 8 & 9 & b & c & d & e & f & g \\
    Demical & 16 & 17 & 18 & 19 & 20 & 21 & 22 & 23 & 24 & 25 & 26 & 27 & 28 & 29 & 30 & 31 \\
    Base32 & h & j & k & m & n & p & q & r & s & t & u & v & w & x & y & z \\
    \bottomrule
    \end{tabular}
\end{table}
% 结束设置三线表

相比于传统的经纬度数据来说,Geohash编码方式将二维查询过程转换为一维字符串匹配,从而在查询性能的提升角度展现出了巨大的优势。Geohash可以实现时间复杂度为O(1)的快速查询,以Geohash编码为基础的空间索引算法针对海量地理数据也同样具有高性能查询能力\cite{2014AGeohash}。

从``北蜂窝路”的示例中可以得出结论,Geohash编码的精度与Geohash编码的位数有关,在本示例中,由于对经纬度各进行15次二分,因此得到的待使用Base32转换的二进制序列一共有30位数据,即对应6位Geohash编码。如果对经纬度各进行20次二分,则得到的待使用Base32转换的二进制序列将一共有40位数据,就会对应8位Geohash编码。该8位Geohash编码相较该6位Geohash的精度更高,且此6位编码与该8位编码的前6位编码数据完全相同,这也意味着一个GeoHash值所代表的区域,一定包含以该GeoHash值为前缀的所有元素。这是Geohash编码最重要的特征。

Geohash编码转换方式在编码长度恰当的情况下,误差是可接受的,如表1-1所示。
% 设置三线表
\begin{table}[ht]
    \linespread{1.5}
    \zihao{5}
    \centering
    \caption{不同长度Geohash编码的误差}
    \label{geo1}
    \begin{tabular}{cccc}
    \toprule
    Geohash length & Lat.err & Lon.err & Km.err \\
    \hline
    1 & ±3 & ±23 & ±2500 \\
    2 & ±2.8 & ±5.6 & ±630 \\
    3 & ±0.7 & ±0.70 & ±78 \\
    4 & ±0.087 & ±0.18 & ±20 \\
    5 & ±0.022 & ±0.022 & ±2.4 \\
    6 & ±0.0027 & ±0.0055 & ±0.61 \\
    7 & ±0.00068 & ±0.00068 & ±0.076 \\
    8 & ±0.000085 & ±0.00017 & ±0.019 \\
    \bottomrule
    \end{tabular}
\end{table}
% 结束设置三线表

Geohash将地理位置根据方格进行划分,适用于车辆调度的相关应用场景。栾方军、张鹏旭、曹科研提出了一种空载出租车巡游路线推荐模型,该模型利用8位长度的Geohash编码,结合地图路网数据,实时预测空载的出租车司机在行驶过程中周边发出打车订单的乘客数量,大幅提高出租车的载客率\cite{栾方军2020Geohash}。

本文课题所依赖的出租车调度系统基于Geohash编码。该系统将Geohash编码方式应用于地图存储与可视化上,并将用Geohash编码方式转换过的车辆与乘客地理位置信息作为区块属性存储,根据地理信息的层次性结构,形成树状区块结构,从而提升在传统区块链中匹配检索地理信息的效率。

\subsection{A*算法}

\space 交互式信息地图的出现,使用户能够即时在线判断自己周围的地理方位,更极大地便利了用户的出行体验。用户在出行时,可以根据交互式信息地图选取最合适自己的出行路径。路径规划方法也应需而生。最广泛的研究领域是最短路径规划,其又分为静态路径规划和动态路径规划。

\space 静态路径规划是在已存在的地理信息、交通规则等约束条件的作用下,寻找最短路径的一类算法。经典的静态路径规划算法有以下几类:

\begin{enumerate}
    \item Dijkstra算法:由Edsger Wybe Dijkstra在1956年提出,用来寻找图形中起点和重点间最短的路径\cite{1959A},即处理单源最短路径。Dijkstra算法以贪心思想为核心,从起点向外层层扩展,直至找到终点,时间复杂度为O(n²)。洪安定将Dijkstra算法与打车平台结合,并投入社会使用,产生了较好的效益\cite{洪安定基于优化司乘匹配的区域网约车平台设计与实现}。
    \item Floyd算法:由Floyd在1962年提出\cite{1962Algorithm},能正确处理有向图、负权的最短路径规划问题,也可以用于计算有向图的传递闭包,即可以处理多源最短路径和负权图,时间复杂度为O(n)。黄文团等基于Floyd算法实现了一个拼车信息服务系统,致力于缓解交通压力,满足城市居民的拼车需求\cite{黄文团2011}。
    \item A*算法:由PE Hart,NJ Nilsson,B Raphael在1968年提出\cite{1972A},它将Dijkstra算法与广度优先算法融合,借助启发式函数,更快找到最优路径,时间复杂度为O(n)。此算法精确稳定,常被使用在各种智能车路径规划场景中。
\end{enumerate}

\space 动态路径规划由Cooke和Halsey于1966年提出\cite{1966The},它根据特定类型的道路权值求解起终点间最短路径,并实时维护道路信息,即时调整预先规划好的最短路径。经典的动态路径规划算法有以下几类:

\begin{enumerate}
    \item DWA算法:由Fox D.和Burgard W.于1997年提出\cite{1997The},它将采样速度设计成一个动态窗口,在该窗口中有很多条可行轨迹,通过自行设计的评价函数评价这些轨迹,挑选出一条最优轨迹,重复迭代搜索当前最优轨迹,从而到达目的地。杨中华等\cite{杨中华0基于改进方向评价函数}对DWA算法的方向评价函数进行改进,为无人车的局部路径规划提供了一条解决方案。
    \item APF算法:由Oussama Khatib在1985年提出\cite{1986Real},它是一种反馈控制策略,通过目标的引理与障碍物的斥力共同引导机器人的移动。杜婉茹\cite{杜婉茹2021面向未知环境及动态障碍的人工势场路径规划算法}等将面向未知复杂环境的动态APF路径规划算法引入进模拟场景中进行仿真,获得了较好的寻路效果。
\end{enumerate}

\space 静态路径规划聚焦于全局规划,动态路径规划聚焦于局部规划,因此二者可以进行结合,协同工作,从而规划出局部最优、全局亦最优的路径。本文所基于的出租车调度系统,采用A*算法进行静态全局寻路,本文将实时路况引入调度系统,使其支持动态路径规划,并获得了较好效果。

\subsection{实时路况}

实时路况是描述一个城市交通道路的拥堵通畅状况的概念,实时路况对交通拥堵的预测具有重大意义\cite{何小波2021基于互联网地图的矢量路况数据生成与应用方法研究}。将实时路况引入出租车调度系统中,能更真实地模拟实际生活中的出租车调度场景,是非常有必要的。

实时路况的采集有以下四类方法\cite{何小波2021基于互联网地图的矢量路况数据生成与应用方法研究}:

\begin{enumerate}
    \item 人工采集:通过人工主动汇报其所在地路况,这种方法时效性低、有效覆盖范围小,费时费力,适合特定路段、特定时间段的路况采集。
    \item 传感器监测:通过道路上安装的摄像头、地感线圈、微波检测器等设备实时监测道路车流量,该方法监测的数据时效性和准确性高,但多由政府、机构掌握,难以共享。
    \item 浮动车法:通过浮动车上安装的GPS设备,实时记录浮动车的位置和速度,该方法监测的数据时效性和准确性高,但数据私有,共享难度大。
    \item 数据众包:通过众包模式获取数据,调用分散的用户各自提交数据,最终获得海量的数据。
\end{enumerate}

浮动车法相比其他几种方式,造价更低、覆盖范围更广,可以提供准确性高的车辆位置和速度信息,因此成为探测城市路况的一种重要数据来源\cite{孙卫真2019低采样率浮动车的路况计算精度优化}。

张禹\cite{张禹基于车辆轨迹的动态路况挖掘}使用基于OHMM的地图匹配算法,将浮动车轨迹与道路相匹配,从而便利路况计算。邱皓月\cite{孙卫真2019低采样率浮动车的路况计算精度优化}采用张禹的道路匹配算法,给出了一种精度更高的低采样率下浮动车的路况计算方法,有效地将实时路况与矢量地图数据进行结合,在考虑历史路况的影响下,使用历史路况对道路的实时路况进行加权平滑,精确推导实时路况。

本文将沿用邱皓月的实时路况计算思路,在本文课题所依赖的出租车调度系统中加入实时路况的影响因素,对属于静态路径规划范畴的A*算法进行优化,实现一个基于实时路况规划路径的出租车调度系统。

\section{本文研究内容、贡献和组织架构}

\subsection{本文研究内容及贡献}

首先,本文研究了传统区块链技术、周畅\cite{s22228885}实现的支持区域划分的树状区块链、智能合约;然后研究了Geohash编码技术的编码方式与特征;接下来研究了静态最短路径规划与动态最短路径规划这两种不同路径规划算法的优缺点与适用场景;最后研究了邱皓月\cite{孙卫真2019低采样率浮动车的路况计算精度优化}实现的实时路况推导过程。

接下来,本文进入数据准备阶段。本文完善了实验室的复现工作,实现了基于树状区块链的出租车调度系统的复现;本文还使用ArcGIS、PostgreSQL软件,结合2023年4月的北京地图数据,按不同Geohash编码划分地图区域,完成了wx4en、wx4er、wx4ep、wx4eq区域里北京真实地图数据的获取,并将地图数据中的经纬度编码为Geohash格式,成功上传到区块链中,更新了出租车调度系统中使用的地图。

紧接着,本文介绍了基于区块链的出租车调度系统的结构。然后本文选取了曼哈顿距离作为A*算法的启发函数,结合公式及场景示例,阐述了A*算法的工作原理及运行流程。在传统A*算法的基础上,本文还基于实时路况提出了结合实时路况的改进A*算法,结合公式及场景实例,阐明了改进A*算法的工作原理与运行流程,并指出了其在出租车调度系统中的实现方法,表明了这种改进A*算法投入应用的可行性。

本文完善的出租车调度系统相比于实验室过往工作中设计的出租车调度系统而言,最大的优势是引入了实时路况的影响因素,从而使该系统能够根据路况动态地规划路径,更接近于真实生活场景。

最后,本文先对地图数据进行了测试,保证上传给区块链的地图数据真实可信,具有可用性;接着,本文把基于实时路况的改进A*算法整合进出租车调度系统中,进行了运行测试与性能测试,证明了将本文所提出的改进A*算法应用在出租车调度系统中这一行为具有可行性。

\subsection{本文组织架构}

本文一共分为五部分,各部分主要内容如下:

第一章介绍了基于区块链的出租车调度系统的背景与意义,并对本文涉及的相关工作、国内外研究现状进行了综述,包括区块链技术和智能合约、Geohash编码、A*算法、实时路况计算,最后介绍了本文的研究内容与贡献。

第二章介绍了本文研究内容前期的数据准备过程,包括对实验室过往工作的复现过程、真实地图数据的处理导入过程。

第三章介绍了基于区块链的出租车调度系统的架构,阐述了A*算法的工作原理和流程,提出了结合实时路况的改进A*算法,设计了基于实时路况规划路径的出租车调度系统。

第四章给出了对本文中获取的地图数据的分析,并对结合了改进A*算法的出租车调度系统进行了运行测试和性能测试。本章给出了完善路径规划算法后的出租车调度系统的整体运行效果和运行数据,并通过对运行数据的分析,证明了改进的A*算法在复杂度方面表现良好,从而证明了把基于实时路况的改进A*算法引入出租车调度系统这一行为具有可行性。

结论总结全文,展望了本文工作在未来仍可拓展的研究发展方向。