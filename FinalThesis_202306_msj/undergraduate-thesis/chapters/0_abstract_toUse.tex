%%
% The BIThesis Template for Bachelor Graduation Thesis
%
% 北京理工大学毕业设计(论文)中英文摘要 —— 使用 XeLaTeX 编译
%
% Copyright 2020-2023 BITNP
%
% This work may be distributed and/or modified under the
% conditions of the LaTeX Project Public License, either version 1.3
% of this license or (at your option) any later version.
% The latest version of this license is in
%   http://www.latex-project.org/lppl.txt
% and version 1.3 or later is part of all distributions of LaTeX
% version 2005/12/01 or later.
%
% This work has the LPPL maintenance status `maintained'.
%
% The Current Maintainer of this work is Feng Kaiyu.

% 中英文摘要章节
\begin{abstract}
% 中文摘要正文从这里开始
汽车出行目前已成为我国社会的交通常态。当下,网约车便于用户预约车辆、实现点对点运输,能满足乘客个性化需求,成为许多人首要的出行方式。但其数据处理、存储均通过第三方平台完成,依赖中心服务器,使数据安全无法得到保证。将区块链与车联网工作结合,保证了司乘间可信交流,提升了打车平台的数据安全性。

本文在实验室前期工作的基础上,调研了树状区块链、Geohash编码、A*算法原理、实时路况计算等技术。本文还复现了实验室前期设计的支持区域划分的树状区块链和出租车调度系统,处理了向区块链上传的北京真实地图数据,从而更新了出租车调度系统中的地图数据。针对实验室已实现的静态路径规划算法,本文提出了基于实时路况的改进A*算法,并将其应用于出租车调度系统。此外,本文对该完善后的出租车调度系统进行了运行测试和性能测试。测试表明,完善后的系统实现了合理的动态路径规划功能,其中改进的A*算法复杂度良好,这证明了本文把基于实时路况的改进A*算法引入出租车调度系统的工作具有可行性。本文工作完善了出租车调度系统的路径规划机制,对真实的出租车调度过程进行了一定程度上的模拟,具有较高实用价值。

\end{abstract}

% 英文摘要章节
\begin{abstractEn}
% 英文摘要正文从这里开始
Car travel has become a normal traffic in our society. At present, online car booking is convenient for users to book a car and realize point-to-point transportation. It can meet the personalized needs of passengers, and has become the primary mode of travel for many people. However, its data processing and storage are completed through the third party platform, relying on the central server, so that the data security can not be guaranteed. The combination of blockchain and Internet of vehicles ensures reliable communication between drivers and passengers, and improves the data security of the ride-hailing platform. 

Based on the previous work in the laboratory, this thesis investigated tree blockchain, Geohash coding technology, A* algorithm principle, real-time traffic situation estimation and other technologies. This thesis reproduces the tree-based blockchain supporting regional division designed by the laboratory earlier, and reproduces the taxi dispatching system based on it. This thesis also processed the real map data of Beijing uploaded to the blockchain, thus updating the map data in the taxi dispatching system. Aiming at the static path planning algorithm realized in the laboratory, this thesis proposes an improved A* algorithm based on real-time traffic situation and applies it to the taxi dispatching system. In addition, this thesis carries on the running test and the performance test of the improved taxi dispatching system. The test results show that the perfected system can realize reasonable dynamic path planning function, and the improved A* algorithm is of good complexity, which proves that it is feasible to introduce the improved A* algorithm based on real-time traffic situation into the taxi dispatching system in this thesis. This thesis improves the route planning mechanism of taxi dispatching system and simulates the real taxi dispatching process to a certain extent, which has high practical value.

\end{abstractEn}
