%%
% The BIThesis Template for Bachelor Graduation Thesis
%
% 北京理工大学毕业设计(论文)结论 —— 使用 XeLaTeX 编译
%
% Copyright 2020-2023 BITNP
%
% This work may be distributed and/or modified under the
% conditions of the LaTeX Project Public License, either version 1.3
% of this license or (at your option) any later version.
% The latest version of this license is in
%   http://www.latex-project.org/lppl.txt
% and version 1.3 or later is part of all distributions of LaTeX
% version 2005/12/01 or later.
%
% This work has the LPPL maintenance status `maintained'.
%
% The Current Maintainer of this work is Feng Kaiyu.
%
% Compile with: xelatex -> biber -> xelatex -> xelatex

\begin{conclusion}
汽车出行目前已成为我国社会的交通常态。相比于公交、地铁、传统打车方式而言,网约车方便用户预约车辆,实现点对点运输,完成对车辆的评价反馈,因此成为许多人首选的出行方式。网约车的兴起,满足了乘客个性化、便捷化的需求。但其数据处理、存储均通过第三方平台完成,依赖中心服务器,使数据安全无法得到保证。将区块链与车联网工作结合,保证司乘间可信交流,提升了打车平台的数据安全性,能营造更加高效且安全的打车服务。但目前已有的研究工作基于传统单链区块链,针对区块链节点在性能方面的处理效果进行优化,缺乏支持区域查询的树状区块链在实际业务应用场景的研究探索。且实验室前期设计实现的出租车调度系统使用静态的路径规划算法,无法模拟真实生活中的出租车调度场景。

针对上述问题,本文首先完成了对树状区块链技术的探索,针对Geohash编码技术与支持区域查询的树状区块链中的地理数据存储方式进行了分析,对传统A*算法与实时路况进行了研究。

接下来,本文基于实验室已有的研究,对已实现的基于传统A*算法的出租车调度系统进行了复现,形成了复现手册。在这一部分,本文还处理获取了较新的北京真实道路数据,将其整合为本系统所需的数据格式,部署到树状区块链上,完成了出租车调度系统中真实地图数据的导入工作。

接着,本文结合实验室早期对实时路况计算的研究,针对传统A*算法提出了基于实时路况的改进A*算法,并将其整合进出租车调度系统中,使车辆运行的路径能够在实时路况的影响下动态更换,从而更加符合真实应用场景。

最后,本文对该基于实时路况的改进A*算法进行了实验探究,以出租车调度系统为载体,分别完成运行实验与性能实验。运行实验证明了本文改进的出租车调度系统能在实时路况的影响下动态规划出更合适的通畅路段,从而说明了本文提出的算法的可运行性;性能实验证明了本文改进的A*算法在复杂度方面表现良好,证明了本文把基于实时路况的改进A*算法引入出租车调度系统的行为具有可行性。

% 不足与展望——
整体而言,本文工作完成了对出租车调度系统中道路规划算法的改进,通过结合北京真实地图数据、引入实时路况,在一定程度上模拟了真实场景中的出租车调度过程,为在基于树状区块链的出租车调度系统中实现完全真实的动态路径规划过程做出了探索,具有较高的实用价值。

但本文工作仍有可扩展的研究方向与不足之处:首先,实时路况计算的过程和方法仍待优化,如将车辆位置定时上传、考虑路口路况计算方式等。此外,通过完善信誉值模块的设计,促进司机与乘客之间的双向评价过程,提升车乘服务质量。最后,可以探索用户的身份认证机制,从而保证准入用户身份的真实性,加强系统安全性。

\end{conclusion}

