%%
% The BIThesis Template for Bachelor Graduation Thesis
%
% 北京理工大学毕业设计(论文)附录 —— 使用 XeLaTeX 编译
%
% Copyright 2020-2023 BITNP
%
% This work may be distributed and/or modified under the
% conditions of the LaTeX Project Public License, either version 1.3
% of this license or (at your option) any later version.
% The latest version of this license is in
%   http://www.latex-project.org/lppl.txt
% and version 1.3 or later is part of all distributions of LaTeX
% version 2005/12/01 or later.
%
% This work has the LPPL maintenance status `maintained'.
%
% The Current Maintainer of this work is Feng Kaiyu.
%
% Compile with: xelatex -> biber -> xelatex -> xelatex

\begin{appendices}
  % 这里示范一下添加多个附录的方法:
  % 使用 \section 来添加一个附录

\section{Geohash编码过程示例}

``北蜂窝路"的经度、纬度表示过程如以下两张表:

\begin{table}[!ht]
    \linespread{1.5}
    % \linespread{1.5}
    \zihao{5}
    \centering
    \caption{经度的二进制表示过程}
    \label{geohash-jingdu}
    \begin{tabularx}{\textwidth}{XXXc}
    \toprule
    纬度 & 划分区间0 & 划分区间1 & 116.3227718 \\
    \hline
    (-180,180) & (-180,0) & (0,180) & 1\\
    (0,180) & (0,90) & (90,180) & 1\\
    (90,180) & (90,135) & (135,180) & 0\\
    (90,135) & (90,112.5) & (112.5,135) & 1\\
    (112.5,135) & (112.5,123.75) & (123.75,135) & 0\\
    (112.5,\newline 123.75) & (112.5,\newline 118.125) & (118.125,\newline 123.75) & 0\\
    (112.5,\newline 118.125) & (112.5,\newline 115.3125) & (115.3125,\newline 118.125) & 1\\
    (115.3125,\newline 118.125) & (115.3125,\newline 116.71875) & (116.71875,\newline 118.125) & 0\\
    (115.3125,\newline 116.71875) & (115.3125,\newline 116.015625) & (116.015625,\newline 116.71875) & 1\\
    (116.015625,\newline 116.71875) & (116.015625,\newline 116.3671875) & (116.3671875,\newline 116.71875) & 0\\
    (116.015625,\newline 116.3671875) & (116.015625,\newline 116.19140625) & (116.19140625,\newline 116.3671875) & 1\\
    (116.19140625,\newline 116.3671875) & (116.19140625,\newline 116.279296875) & (116.279296875,\newline 116.3671875) & 1\\
    (116.279296875,\newline 116.3671875) & (116.279296875,\newline 116.3232421875) & (116.3232421875,\newline 116.3671875) & 0\\
    (116.279296875,\newline 116.3232421875) & (116.279296875,\newline 116.30126953125) & (116.30126953125,\newline 116.3232421875) & 1\\
    (116.30126953125,\newline 116.3232421875) & (116.30126953125,\newline 116.312255859375) & (116.312255859375,\newline 116.3232421875) & 1\\
    \bottomrule
    \end{tabularx}
\end{table}

% 设置三线表
\begin{table}[!ht]
    % \linespread{1.4}
    \linespread{1.5}
    \zihao{5}
    \centering
    \caption{纬度的二进制表示过程}
    \label{geohash-weidu}
    \begin{tabularx}{\textwidth}{XXXc}
    \toprule
    纬度 & 划分区间0 & 划分区间1 & 39.9037387 \\
    \hline
    (-90,90) & (-90,0) & (0,90) & 1 \\
    (0,90) & (0,45) & (45,90) & 0 \\
    (0,45) & (0,22.5) & (22.5,45) & 1 \\
    (22.5,45) & (22.5,33.75) & (33.75,45) & 1 \\
    (33.75,\newline 45) & (33.75,\newline 39.375) & (39.375,\newline 45) & 1 \\
    (39.375,\newline 45) & (39.375,\newline 42.1875) & (42.1875,\newline 45) & 0 \\
    (39.375,\newline 42.1875) & (39.375,\newline 40.78125) & (40.78125,\newline 42.1875) & 0 \\
    (39.375,\newline 40.78125) & (39.375,\newline 40.078125) & (40.078125,\newline 40.78125) & 0 \\
    (39.375,\newline 40.078125) & (39.375,\newline 39.7265625) & (39.7265625,\newline 40.078125) & 1 \\
    (39.7265625,\newline 40.078125) & (39.7265625,\newline 39.90234375) & (39.90234375,\newline 40.078125) & 1 \\
    (39.90234375,\newline 40.078125) & (39.90234375,\newline 39.990234375) & (39.990234375,\newline 40.078125) & 0 \\
    (39.90234375,\newline 39.990234375) & (39.90234375,\newline 39.9462890625) & (39.9462890625,\newline 39.990234375) & 0 \\
    (39.90234375,\newline 39.9462890625) & (39.90234375,\newline 39.92431640625) & (39.92431640625,\newline 39.9462890625) & 0 \\
    (39.90234375,\newline 39.92431640625) & (39.90234375,\newline 39.913330078125) & (39.913330078125,\newline 39.92431640625) & 0 \\
    (39.90234375,\newline 39.913330078125) & (39.90234375,\newline 39.9078369140625) & (39.9078369140625,\newline 39.913330078125) & 0 \\
    \bottomrule
    \end{tabularx}
\end{table}
% 结束设置三线表


\newpage

\section{使用SQL语句从postgreSQL中导出地图数据}

当已完成对表\ref{mapData_Addproperty}中的内容添加过程后,需要使用以下语句从数据库中将地图数据导出。该语句本质上是一套查询语句的嵌套过程,先按照筛选要求查询得到n条道路数据,再将这n条道路数据各自组合成n条区块链上所需的Json格式的数据,最后将这n条Json语句组合成一条Json语句。以下SQL代码段以导出wx4eq区域Json格式的地图数据为示例。

\begin{lstlisting}[language=sql, caption={导出wx4eq区域Json格式的地图数据}, label={lst:getmapdata}]
SELECT
  array_to_json(array_agg(row_to_json(ff)))
FROM
  (SELECT
    ( SELECT row_to_json(t) FROM
      ( select minzoom, fclass as highway, round(cost::numeric,0) as cost, gid, name, source, target, one_way as oneway ) 
      AS t 
    ) AS properties ,
    ST_AsGeoJSON(geom)::json AS geometry,
    'Feature' AS type 
  FROM bjway
  WHERE
    ((code = 5112 or code = 5113 or code = 5114 or code = 5115 or code = 5132 or code = 5133 or code = 5134 or code = 5135) and (start_x between 116.279 and 116.323) and (start_y between 39.947 and 39.991) and oneway = 'F') 
  order by gid) 
AS ff;\end{lstlisting}

\newpage

\section{方格地图上,传统A*算法规划的路径}
下表给出方格地图上,使用传统A*算法在出租车调度系统内运行出的路径规划结果。其中车辆初始位置为wx4erjmbekd,乘客起点为wx4erxjzekd,乘客终点为wx4erjmbekd。
% 设置三线表
\begin{table}[!ht]
    \linespread{1.5}
    \zihao{5}
    \centering
    \caption{传统A*算法给出的路径规划结果}
    \label{exper1}
    \begin{tabularx}{\textwidth}{p{3cm}X}
    \toprule
    阶段 & 路径规划结果 \\
    \hline
    pickUp() & wx4erjmbekd, wx4erjjzekd, wx4erjnpekd, wx4erjppekd\newline wx4erm0zekd, wx4erm1zekd, wx4erm4pekd, wx4erm5pekd\newline wx4ermhzekd, wx4ermjzekd, wx4ermnpekd, wx4ermppekd\newline wx4ert0zekd, wx4ert1zekd, wx4ert4pekd, wx4ert60ekd\newline wx4ertdpekd, wx4ertf0ekd, wx4erw4pekd, wx4erw60ekd\newline wx4erwdpekd, wx4erwf0ekd, wx4erx4pekd, wx4erx5pekd\newline wx4erxhzekd, wx4erxjzekd \\
    manageToEnd() &  wx4erxjzekd, wx4erwvbekd, wx4erwubekd, wx4erwg0ekd\newline wx4erwf0ekd, wx4erwcbekd, wx4erwbbekd, wx4erqz0ekd\newline wx4erqy0ekd, wx4erqvbekd, wx4erqubekd, wx4erqg0ekd\newline wx4erqf0ekd, wx4erqcbekd, wx4erqbbekd, wx4ernz0ekd\newline wx4erny0ekd, wx4ernvbekd, wx4erntzekd, wx4ernmbekd\newline wx4ernjzekd, wx4erjvbekd, wx4erjtzekd, wx4erjmbekd \\
    \bottomrule
    \end{tabularx}
\end{table}
% 结束设置三线表

在传统A*算法的控制下,只要地图不变,车辆位置、乘客起点与终点不变,无论系统运行多少次,规划出的路线都是上述路线,且该路线对应的总距离成本永远不变。

\newpage

\section{Github地址索引}

本文工作对应的Github仓库地址为:

\href{https://github.com/LancerEnk/GraduationDesign}{https://github.com/LancerEnk/GraduationDesign}

本文中提及的文件所在的仓库链接如下:
\begin{enumerate}
    \item \href{https://github.com/LancerEnk/GraduationDesign/blob/main/doc/\%E5\%A4\%8D\%E7\%8E\%B0\%E6\%89\%8B\%E5\%86\%8C/10\%20\%E9\%83\%A8\%E7\%BD\%B2\%E5\%9C\%A8geth-tree\%E4\%B8\%8A\%E7\%9A\%84\%E5\%87\%BA\%E7\%A7\%9F\%E8\%BD\%A6\%E8\%B0\%83\%E5\%BA\%A6\%E7\%B3\%BB\%E7\%BB\%9F\%E5\%A4\%8D\%E7\%8E\%B0\%E5\%AE\%9E\%E9\%AA\%8C.md}{第2章:树状区块链复现手册地址}
    \item \href{https://github.com/LancerEnk/GraduationDesign/blob/main/doc/\%E5\%A4\%8D\%E7\%8E\%B0\%E6\%89\%8B\%E5\%86\%8C/vue\%E5\%A4\%8D\%E7\%8E\%B0.md}{第2章:vue前端复现手册地址}
    \item \href{https://github.com/LancerEnk/GraduationDesign/tree/main/fuxian/mapData/RealBjMap}{第2章:真实北京地图数据文件地址}
    \item \href{https://github.com/LancerEnk/GraduationDesign/tree/main/src/0521data}{第4章:综合实验的数据文件地址}
\end{enumerate}

\end{appendices}